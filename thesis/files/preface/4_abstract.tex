\cleardoublepage
\phantomsection
\pdfbookmark{Compendio}{Compendio}
\begingroup
\let\clearpage\relax
\let\cleardoublepage\relax
\chapter*{Sommario}

Il presente documento descrive il lavoro svolto dal laureando Dal Bianco Riccardo durante il periodo di \textit{stage} svolto presso l'azienda VisioneImpresa.
La relazione di fine \textit{stage} si divide in quattro capitoli:
\begin{enumerate}
		\item Presentazione dell'azienda, informazioni generali su VisioneImpresa, i suoi prodotti, tecnologie utilizzate e propensione all'innovazione.
		\item  Contestualizzazione del progetto di \textit{stage} all'interno della strategia aziendale, perché è stato proposto e aspettative aziendali e personali.
		\item Descrizione delle attività svolte durante il periodo di \textit{stage}: l'analisi, la progettazione, parte dello sviluppo, la validazione e i risultati conseguiti.
		\item Raggiungimento degli obiettivi prefissati, valutazione e retrospettiva del periodo di \textit{stage} e del percorso universitario.
\end{enumerate}

\textbf{Convenzioni tipografiche} \\
All'interno del documento vengono utilizzate le seguenti convenzioni tipografiche:
\begin{itemize}
	\item \textit{Corsivo} indica i termini in lingua non italiana.
	\item \textbf{Grassetto} utilizzato negli elenchi puntati per enfatizzare la parola chiave e utilizzato nei titoli per enfatizzare le porzioni di testo.
	\item \texttt{Monospaziato} utilizzato per indicare nomi di funzioni, \textit{database}, tabelle, \textit{file} e classi.
	\item Parole del Glossario, rappresentate in corsivo, in blu e da una G a pedice (ad esempio \mygls{ERP}).
\end{itemize}

\endgroup
\vfill
