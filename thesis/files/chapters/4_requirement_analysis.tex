\chapter{Valutazione retrospettiva}
\label{chap:Valutazione-Retrospettiva}

\section{Valutazione degli obiettivi}
\subsection{Obiettivi  aziendali}
Gli obiettivi aziendali rivolti al progetto di \textit{stage} erano diversi e distribuiti sui due moduli differenti che ne facevano parte. Come indicato all'interno della tabella \ref{tab:obiettivi-aziendali} e \ref{tab:obiettivi-aziendali-interventi}, gli obiettivi per questi progetti erano divisi in obbligatori, desiderabili e facoltativi.

Nel caso del modulo legato all'estensione del raggio di competenza della \textit{chatbot} aziendale chiamata VisionAI, ho raggiunto pienamente i due obiettivi obbligatori:
\begin{itemize}
	\item O001: sviluppo di interfacce con DevExpress - l'interfaccia Windows \textit{form} è stata implementata con successo, contenendo tutte le funzionalità richieste come connessione diretta alla \textit{chat} salvando i dati e le sessioni e modificare e aggiungere connessioni ai \textit{database} 
	\item O002: sviluppo \mygls{API} \mygls{REST} in C\# - ho contribuito attivamente allo sviluppo delle nuove \mygls{API} ma anche alla correzione di quelle esistenti, al fine di estendere la funzionalità dell'assistente VisionAI. Tutte le \mygls{API} le ho  sviluppate seguendo l'architettura definita dall'azienda utilizzando ASP.NET Core. 
\end{itemize}
Ho raggiunto anche l'obiettivo D01 che comprendeva la documentazione e il piano dei \textit{test}, seppur in parte, vista l'assenza di una \textit{suite} di \textit{test} automatici, compensato però da una rigida verifica manuale delle funzionalità e una documentazione dettagliata.
Per quanto riguarda il modulo legato al contesto di VisionAssistance, l'obiettivo indicato dall'azienda era facoltativo, F01 ovvero la creazione di un \textit{agent} per la pianificazione automatica degli interventi, questo obiettivo è stato raggiunto sotto forma di \mygls{PoC} che ha dimostrato la fattibilità del progetto, rispettando i vincoli imposti e dando risposte coerenti.

In generale ho realizzato tutti gli obiettivi aziendali, ottenendo anche un buono stato di soddisfacimento da parte del referente aziendale al momento della presentazione del lavoro svolto.

\subsection{Obiettivi  personali}
Prima che iniziasse lo \textit{stage}, mi ero posto alcuni obiettivi personali con l'intenzione di sfruttare l'esperienza per apprendere molto sia dal un punto di vista formativo che umano.
Gli obiettivi che mi sono posti non erano riguardanti solo l'acquisizione di competenze tecniche, ma anche il modo in cui volevo affrontare il lavoro in un contesto aziendale.

Uno degli aspetti riguardava la volontà di apprendere nuove tecnologie, in particolare legate all'ambiente di .NET e allo sviluppo in linguaggio C\#. Provenendo da un ambiente accademico incentrato molto sulla teoria mi interessava  ampliare il mio \textit{stack} tecnologico con soluzioni legate prettamente al mondo del lavoro. Lavorare su un progetto reale utilizzando queste tecnologie mi ha permesso di crescere molto.
Un altro obiettivo importante per me era quello di sviluppare un metodo di lavoro, imparando a pianificare le attività rispettare le scadenze e avere un flusso di lavoro continuo e positivo. Volevo seguire una struttura ben definita organizzandomi le cose da fare e arrivando a terminare il lavoro senza ritardi e difficoltà e il contesto aziendale si è rivelato molto utile per raggiungere questo obiettivo.
Avevo inoltre la volontà di rafforzare le mie capacità relazionali, sia all'interno del \textit{team} di sviluppo sia nel confronto con altri membri dell'azienda. La possibilità di collaborare al progetto è stata molto importante, permettendomi di lavorare sempre insieme a qualcuno migliorando molto le mie capacità comunicative e rendendo il progetto più interessante.
Infine uno degli aspetti che ritenevo più importanti era quello di mettermi alla prova in un ambiente lavorativo, allontanandomi dalla zona di \textit{comfort} dell'università. Ho cercato di affrontare con positività nuove situazioni e a valorizzare questa esperienza come possibilità di crescita.
Per concludere posso confermare che gli obiettivi personali sono stati raggiunti in modo pieno, ho trovato questa esperienza molto formativa sotto sia il punto di vista tecnico che umano.


\section{Conoscenze e competenze acquisite}
Durante il periodo di \textit{stage} ho avuto la possibilità di imparare moltissimo accrescendo il mio bagaglio tecnico e di competenze trasversali, l'esperienza è stata anche più ricca di quanto mi aspettassi poiché mi ha permesso di lavorare su progetti complessi e concreti e con tecnologie utilizzate nel mondo del lavoro.
Uno degli aspetti più importanti se parliamo di competenze acquisite è sicuramente l'ambiente .NET e il linguaggio C\#, che ho utilizzato in maniera quotidiana per lo sviluppo del \textit{backend} di VisionAI. Pur avendo qualche nozione di base durante lo \textit{stage} ho appreso l'organizzazione di un vero progetto aziendale con un architettura a strati e l'importanza di alcuni concetti base per la scrittura di un buon codice. In particolare ho appreso familiarità con lo sviluppo di \mygls{API} tramite ASP.NET Core, la progettazione di un sistema \textit{controller, repository e service} e l'utilizzo di tecnologie e \textit{framework} correlate all'ambiente .NET.
Ho imparato molto anche dal punto di vista del lavoro con \textit{database}, lo \textit{stage} ha rappresentato proprio un salto di livello. Pur partendo con una base di nozioni solida sia per quanto riguarda SQL sia per quanto riguarda il \textit{database}, confrontarmi con una base di dati complessa e un sistema di grandi dimensioni basato sulle relazioni tra tabelle mi ha portato a sviluppare competenze più avanzate. Sicuramente ho imparato a scrivere \textit{query} complesse, lavorare con un \textit{database} di grandi dimensioni comprendendone la struttura e utilizzare SQL Server Management Studio come strumento quotidiano.
Oltre a questi ho potuto utilizzare molti altri strumenti come Google Or-Tools e apprndere nozioni sulla programmazione vincolata, utilizzare \mygls{ORM} come Entity framework e SQLAlchemy conoscendo molte nuove tecnologie e soluzioni.
Oltre alle competenze tecniche lo \textit{stage} mi ha permesso di crescere anche con abilità trasversali fondamentali per il mondo del lavoro, come il lavorare in \textit{team} in maniera continuativa, confrontandomi ogni giorno con un altra persona e condividendo idee e problemi. Migliorare sotto il punto di vista comunicativo e di organizzazione del lavoro, relazionarmi con un ambiente lavorativo e tutte le dinamiche interne. Tutte queste competenze e conoscenze che ho appreso in questo periodo le reputo molto preziose.

\section{Considerazioni finali sul percorso universitario}
Il percorso universitario che ho affrontato è stato sicuramente impegnativo e con molti ostacoli, allo stesso tempo però molto stimolante e costruttivo. Questi anni all'interno dell'ambiente universitario sono stati molto formativi non solo dal punto di vista di apprendimento ma anche dal punto di vista metodologico e umano, e tornassi indietro sceglierei ancora questo percorso consapevole del valore che mi ha restituito.
Durante questo periodo, lavorando all'interno di un'azienda e partecipando in maniera attiva a progetti veri mi sono reso conto che l'università ti prepara al mondo del lavoro, anche se in un modo meno diretto di quello che si può pensare. Non ti insegna ad usare ogni strumento o a conoscere ogni linguaggio, sarebbe impossibile, soprattutto nel mondo dell'informatica che è in continua evoluzione.
Quello che invece ti offre è una base molto solida di conoscenze teoriche e un metodo di ragionamento. Ti insegna come imparare, come risolvere problemi e a risolverli da solo.
Con questa esperienza ho capito in modo significativo di quanto le basi siano essenziali, perché senza fondamenta non si costruisce niente di solido.
In informatica imparare come affrontare un problema, anche senza conoscere a priori la tecnologia, è più importante che conoscere la tecnologia stessa e questo è alla base di questa università.


\newpage


