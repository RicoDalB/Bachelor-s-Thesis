\cleardoublepage
\appendix
\pagenumbering{roman} % Imposta numeri romani PRIMA di settare il numero
\setcounter{page}{1}  % Solo se vuoi che il Glossario inizi da i (opzionale)
\phantomsection
\addcontentsline{toc}{chapter}{Glossario}
\chapter*{Glossario}


\begin{description}
  \item[\hypertarget{gls:API}{API}] (Application Program Interface) In un programma informatico per \textit{API} si intende un insieme di definizioni o protocolli che permettono a componenti \textit{software} di comunicare con altri componenti \textit{software} e scambiarsi dati. Le \textit{API} permettono la condivisione solo dei dati necessari mantenendo nascosti i dettagli interni del sistema. 

  \item[\hypertarget{gls:ERP}{ERP}] Un \textit{enterprise resource planning} è un tipo di \textit{software} di gestione che integra tutti i processi aziendali come vendite, acquisti, gestione magazzino, finanza e contabilità. I sistemi \textit{ERP} centralizzano i dati provenienti da un insieme di processo di \textit{business} rendendoli disponibili a tutti. Oggi i sistemi \textit{ERP} risultano fondamentali per la gestione delle aziende.
  
  \item[\hypertarget{gls:PBX}{PBX}] Un \textit{private branch exchange} nelle telecomunicazioni è una rete telefonica privata che viene usata all'interno di un azienda o organizzazione. Questa infrastruttura permetta la comunicazione interna ed esterna. Un \textit{PBX} consente di avere più telefoni rispetto alle linee telefoniche fisiche e consente chiamate gratuite tra gli utenti. 
  
  \item[\hypertarget{gls:IDE}{IDE}] Con \textit{integrated development environment} o ambiente di sviluppo integrato si intende un \textit{software} progettato per la realizzazione di applicazioni che riunisce diversi strumenti di sviluppo in un unica interfaccia grafica. Costituito da un \textit{editor} di codice sorgente, un automazione della \textit{build} locale e un \textit{debugger}. 
  
  \item[\hypertarget{gls:REST}{REST}] \textit{Representational state transfer} si tratta di un architettura\textit{software} che impone condizioni sul funzionamento di un \textit{API}.  Si utilizza l'architettura basata su \textit{REST} per supportare comunicazioni affidabili e con alte prestazioni su larga scala. Gli sviluppatori di \textit{API} possono progettarle utilizzando diverse architetture e quelle che seguono le architetture \textit{REST} sono chiamate \textit{RESTful}.
  
  \item[\hypertarget{gls:ORM}{ORM}] \textit{Object-relational mapping} nell'informatica è un \textit{pattern} di programmazione che crea un ponte virtuale tra tra il paradigma orientato agli oggetti e il modello relazionale dei \textit{database}, permettendo agli sviluppatori di manipolare dati ed effettuare chiamate utilizzando gli oggetti nel linguaggio di programmazione corrente senza inserire codice in SQL. 
  
  \item[\hypertarget{gls:DTO}{DTO}] \textit{data transfer object} è un oggetto che definisce le modalità di invio dati in rete o tra sottoinsiemi di un applicazione software, rendendo possibile mappare i dati che si voglio trasferire nascondendo quelli non opportuni o sensibili. 
  
  \item[\hypertarget{gls:HTTP}{HTTP}] \textit{Hypertext Transfer Protocol} è un protocollo di comunicazione alla base della trasmissione di dati sul \textit{web}, e definisce i comandi e i servizi utilizzati per il trasferimento dei dati in rete.
   
  
  \item[\hypertarget{gls:LLM}{LLM}] \textit{large language model}, in italiano modello linguistico di grandi dimensioni,è una tecnologia molto avanza nell'ambito \textit{AI}, incentrata sulla comprensione, l'analisi e la generazione del testo in ambito generale.
   
  
  \item[\hypertarget{gls:PoC}{PoC}] \textit{proof of concept} ha lo scopo di determinare la fattibilità dell'idea e verificare che l'idea funzionerà come previsto. Può essere visto come una fase di \textit{test} iniziale dell'idea e come un prototipo.

  
  \item[\hypertarget{gls:legacy}{Legacy}]  Un sistema \textit{legacy} in informatica è un sistema utilizzati per un periodo prolungato che generalmente presenta una tecnologia obsoleta, prestazioni inefficienti, vulnerabilità di sicurezza, costi di manutenzione elevati, scalabilità limitata e scarsa attendibilità.
  
  
  \item[\hypertarget{gls:endpoint}{Endpoint}] un \textit{endpoint} è un luogo digitale in cui un \textit{API} riceve chiamate, note come richieste, per le risorse sul suo \textit{server}. Gli \textit{endpoint} sono complementari alle \textit{API} e si presentano molto spesso sotto forma di \textit{URL}.
  
  
  \item[\hypertarget{gls:Clean architecture}{Clean architecture}] Il termine \textit{Clean architecture} deriva dall'omonimo libro di Robert C. Martin e possiamo pensarla come un insieme di linee guida per progettare l'architettura di un \textit{software}. I suoi principi si possono riassumere in indipendenza dal \textit{framework}, indipendenza dall'\textit{UI}, indipendenza dal \textit{database} e \textit{testabile}.
  
\end{description}

\cleardoublepage
\pagenumbering{roman}
